The Dumbbell Nebula (also known as the Apple Core Nebula, Messier 27, and NGC 6853) is a planetary nebula (nebulosity surrounding a white dwarf) in the constellation Vulpecula, at a distance of about 1360 light-years.[1] It was the first such nebula to be discovered, by Charles Messier in 1764. At its brightness of visual magnitude 7.5 and diameter of about 8 arcminutes, it is easily visible in binoculars[4] and is a popular observing target in amateur telescopes. The Dumbbell Nebula appears shaped like a prolate spheroid and is viewed from our perspective along the plane of its equator. In 1992, Moreno-Corral et al. computed that its rate of expansion angularly was, viewed from our distance, no more than 2.3 arcseconds (″) per century. From this, an upper limit to the age of 14,600 years may be determined. In 1970, Bohuski, Smith, and Weedman found an expansion velocity of 31 km/s. Given its semi-minor axis radius of 1.01 ly, this implies that the kinematic age of the nebula is 9,800 years.[3][5]
