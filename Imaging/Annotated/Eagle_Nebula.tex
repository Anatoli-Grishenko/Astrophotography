The Eagle Nebula (catalogued as Messier 16 or M16, and as NGC 6611, and also known as the Star Queen Nebula) is a young open cluster of stars in the constellation Serpens, discovered by Jean-Philippe de Cheseaux in 1745–46. Both the "Eagle" and the "Star Queen" refer to visual impressions of the dark silhouette near the center of the nebula,[4][5] an area made famous as the "Pillars of Creation" imaged by the Hubble Space Telescope. The nebula contains several active star-forming gas and dust regions, including the aforementioned Pillars of Creation. The Eagle Nebula lies in the Sagittarius Arm of the Milky Way.
